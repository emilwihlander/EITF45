% !TeX spellcheck = sv_SE
\documentclass[a4paper]{IEEEtran}
\def\thepage{} %a4paper adds page numbers, this fix removes them

\usepackage[pdftex]{graphicx}
\usepackage[T1]{fontenc}
\usepackage[utf8]{inputenc} 

\usepackage[swedish]{babel}
\usepackage{url}

\usepackage{scrextend}

\title{Autonomous Vehicles}

% 1. Teknisk översikt
% 2. Nutida tillämplingar
% 3. Framtida och möjliga tillämpningar
% 4. Konkurrerande teknologier och standarder. Fördelar och nackdelar
% 5. Egna slutsatser

\author{\IEEEauthorblockN{Niklas Hedström, Emil Wihlander\\ }
\IEEEauthorblockA{Lunds Tekniska Högskola\\
Lund, Sverige\\
Email: \{dat15ewi, dat15nhe\}@student.lu.se}}

%----------------------------------------------------------------

\begin{document}
\maketitle

\begin{abstract}

\end{abstract}


\section{Introduktion}
\emph{Autonomous vehicles}, eller \emph{självkörande fordon}, är fordon som på ett eller annat sätt kan styra sig själv baserat på omgivningen. \emph{SAE International} har definierat klassificeringsnivåer som inom industrin har blivit allmänt accepterade där fordon klassas från SAE level 0 - Ingen automatisering, till SAE level 5 - fullt automatiserad. \cite{SAE2014} 

Huvudfokus kommer ligga på \emph{självkörande bilar} snarare fordon där det finns ett stort intresse både bland klassiska biltillverkare så som Volvo, Mercedes-Benz och Ford och nya företag inom bilbranschen så som Google, Tesla och Uber.\cite{VolvoAD}\cite{MercedesAD}\cite{FordAD}\cite{GoogleAD}\cite{TeslaAD}\cite{UberAD}

\section{Klassificering}
\emph{SAE international standard J3016} definierar de olika nivåerna av självförning enligt\cite{SAE2014}:

\vspace{10 pt}
\begin{labeling}{nivå 1}
	\item [\textbf{nivå 0}] \emph{No Automation}. Fordonet saknar helt självkörning. Kan skicka varningar till föraren, men är inget krav.
	\item [\textbf{nivå 1}] \emph{Driver Assistance}. Fordonet har vissa funktioner som påverkar det baserat på omgivningen. T.ex. ACC (Adaptive Cruise Control)\footnote{När fordonet kan ändra farthållaren baserat på hastigheten av framförvarande fordon\cite{ACC}}, LKA (Lane Keeping Assistance)\footnote{När fordonet kan hjälpa till att styra så att den håller sig inom nuvarande fil\cite{LKA}} och Parkeringshjälp\footnote{När fordonet hjälper till att parkera genom att ta över styrningen\cite{AP}}. Föraren måste dock alltid vara redo att ta över.
	\item [\textbf{nivå 2}] \emph{Partial Automation}. Fordonet kan själv manövrera sig i kända förutsättningar, men när förutsättningar inte längre uppfylls måste föraren ta över genast.
	\item [\textbf{nivå 3}] \emph{Conditional Automation}. Fordonet ska utöver nivå 2 kunna hantera dynamiska situationer i specifika miljöer, så som huvudleder där gångtrafikanter saknas. Detta innebär att föraren kan släppa fokus helt i dessa miljöer.  
	\item [\textbf{nivå 4}] \emph{High Automation}. Fordonet ska utöver nivå 3 kunna hantera situationer som inte förväntas uppstå och kunna agera därefter.
	\item [\textbf{nivå 5}] \emph{Full Automation}. Fordonet ska utöver nivå 4 kunna hantera alla miljöer och därmed aldrig kräva input från en potentiell förare.
\end{labeling}

%\subsection{Ett underavsnitt}
\section{Controller Area Network}

\subsection{Physical Layer}

\subsection{Data Link Layer}

\section{FlexRay}%?

\section{Vehicle Area Network}%?

\section{Car to Car Communication}

\section{Cloud to Car Communication}

\section{Sammanfattning}



% figure section to copy
%\begin{figure}
%    \begin{center}
%        \resizebox{!}{40mm}{\includegraphics{example.pdf}}
%    \end{center}
%    \caption{Conceptual network diagram showing the DSL network.}
%    \label{DSLsetup}
%\end{figure}




\begin{thebibliography}{77}
	\bibitem{SAE2014}SAE International: \emph{Automated Driving}, 
	
	http://www.sae.org/misc/pdfs/automated\_driving.pdf, 
	
	2014 (hämtad 2016-12-02)
	
	\bibitem{VolvoAD}Volvo: \emph{Volvo Cars presents a unique solution for integrating self-driving cars into real traffic}, 
	
	https://www.media.volvocars.com/global/en-gb/media/pressreleases/158276/volvo-cars-presents-a-unique-system-solution-for-integrating-self-driving-cars-into-real-traffic, 
	
	2015-02-19 (hämtad 2016-12-02)
	
	\bibitem{MercedesAD}Mercedes: \emph{The Mercedes-Benz F 015 Luxury in Motion.}, 
	
	https://www.mercedes-benz.com/en/mercedes-benz/innovation/research-vehicle-f-015-luxury-in-motion/ 
	
	(hämtad 2016-12-02)
	
	\bibitem{FordAD}Ford: \emph{Ford börjar testa självkörande bilar i Europa under 2017}, 
	
	http://www.mynewsdesk.com/se/ford/pressreleases/ford-boerjar-testa-sjaelvkoerande-bilar-i-europa-under-2017-1670717, 
	
	2016-11-29 (Hämtad 2016-12-02)
	
	\bibitem{GoogleAD}Google: \emph{Google Self-Driving Car Project}, 
	
	https://www.google.com/selfdrivingcar/ 
	
	(hämtad 2016-12-02)
	
	\bibitem{TeslaAD}Tesla: \emph{All Tesla Cars Being Produced Now Have Full Self-Driving Hardware}, 
	
	https://www.tesla.com/blog/all-tesla-cars-being-produced-now-have-full-self-driving-hardware, 
	
	2016-10-19 (hämtad 2016-12-02)
	
	\bibitem{UberAD}Uber: \emph{Pittsburgh, your Self-Driving Uber is arriving now}, 
	
	https://newsroom.uber.com/pittsburgh-self-driving-uber/, 
	
	2016-09-14 (hämtad 2016-12-02)
	
	\bibitem{ACC}Wikipedia: \emph{Autonomous cruise control}, 
	
	https://en.wikipedia.org/wiki/Autonomous\_cruise\_control\_system, 
	
	2016-12-01 (Hämtad 2016-12-01)
	
	\bibitem{LKA}Toyota: \emph{Lane Keeping Assist}, 
	
	http://www.toyota-global.com/innovation/safety\_technology/safety\_
	
	technology/technology\_file/active/lka.html 
	
	(Hämtad 2016-12-01)
	
	\bibitem{AP}Wikipedia: \emph{Automatic parking}, 
	
	https://en.wikipedia.org/wiki/Automatic\_parking, 
	
	2016-11-24 (Hämtad 2016-12-01)
%\bibitem{gsma}S. Parkvall: \emph{Broadband Wireless Access - HSPA and LTE}, http://www.s3.kth.se/signal/edu/s3\_seminar/2009/talks/sem2.pdf (last visited 2009-03-12)
%\bibitem{hostmfl} P. Ödling, T. Magesacher, M. Berg, E. A. Sanchez, S. Höst and P. Börjesson: \emph{The Fourth Generation Broadband Concept}, IEEE Communications Magazine, Vol. 47, No. 1, pp. 63-69, IEEE Communication Society, 2009
\end{thebibliography}

\end{document}
